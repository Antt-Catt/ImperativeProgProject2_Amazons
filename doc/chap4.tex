\chapter{Conclusion}

\section{Difficultés rencontrées}

Lors de ce projet, certaines difficultés ont su freiner notre progression et nous faire réfléchir, notamment la découverte et le développement d'un jeu sous forme de \textbf{client/serveur} qui au départ semblait complexe. L'implémentation du monde a également posé problème au debut du projet car les pistes de réflextions sur comment faire n'aboutissait pas. Cependant les matrices GSL ont été le plus gros frein. Ces dernières étant nouvelles pour le groupe leurs utilisation n'était pas chose aisée bien qu'une documentation soit à notre disposition. L'extraction et la configuration de valeurs dans ces matrises nous a posé problème pendant un certain temps.    

\section{Bilan du projet}
Au final, le rendu est un jeu  des amazones avec un plateau classique pouvant aller jusqu'à plus de $150 \times 150$. Ce plateau est représenté à l'aide d'un graphe de liaison qui n'est pas modifié au cours de la partie. Sur ce plateau, deux joueurs peuvent s'affronter avec chacun une stratégie différente et une représentation du monde qui leur est propre. Une partie serveur est également présente afin de lier tout ça et de vérifier le bon déroulement de la partie. Ainsi, à la fin de chaque partie, un vainqueur est désigné et les joueurs créés en local en exportant leur bibliothèque peuvent aller affronter d'autres joueurs appartenant aux autres équipes de projet. 
\\
Cependant, le projet est loin d'être complet et voici quelques idées de points à améliorer ou à implémenter :
\medbreak
\begin{itemize}
    \item \textbf{Types de monde}: Différents types de monde nous ont été proposés et nous n'avons eu le temps d'en réaliser qu'un seul qui fonctionne correctement.
    \item \textbf{Stratégie}: Ce point peut être amélioré de beaucoup de façons différentes, et la meilleure façon en développant une sorte d'IA qui choisirait à chaque fois le meilleur coup possible pour arriver à vaincre l'adversaire.
    \item \textbf{Formation de départ}: Dans l'optique cette fois-ci de modifier un petit peu le jeu, de nouvelles formations de départ pourrait être une idée de petit changement à implémenter.
\end{itemize}

\medbreak

Pour conclure, ce projet aura permis d'avoir une nouvelle approche sur le fonctionnement et le développement d'un jeu de plateau qui se voudrait multijoueurs. Mais également aura ouvert la réflexion et fait découvrir le fonctionnement de l'IA et des différentes méthodes de maximisations des chances pour jouer le meilleur coup grâce à l'intervention des deuxièmes années.



